\documentclass[aspectratio=169,xcolor=table]{beamer}
\usepackage[utf8]{inputenc}
\usepackage[T1]{fontenc}
\usepackage{lmodern}
\usepackage{csquotes}
\usepackage{xcolor}
\usepackage[portuguese]{babel}
\usepackage{hyperref}
\usepackage{tabularx}
\usepackage{amsmath}
\usepackage{amssymb}
\usepackage{tikz}
\usetikzlibrary{shapes.geometric, arrows.meta, positioning, fit, calc, mindmap, shadows, matrix}

% ------------------------------------------------
% Tema e Configurações do Beamer
% ------------------------------------------------
\usetheme{DCC}

% Ajuste de margens para evitar overlap com sidebar
\setbeamersize{text margin left=2.2cm, text margin right=1.0cm}

% Ajuste de espaçamento entre itens
\setbeamertemplate{itemize items}[circle]
\setbeamertemplate{itemize subitem}[circle]
\setlength{\itemsep}{0.6em}
\setlength{\parskip}{0.4em}

\graphicspath{{imgs/}{./imgs/}}

\author[Luis Felipe Sena]{%
  \textbf{Luis Felipe Cordeiro Sena}\\
  \small Orientador: Prof. Frederico Araújo Durão
}
\title{Sistema de Monitoria-IC}
\subtitle{Plataforma Web para Gestão Completa de Monitorias Acadêmicas da UFBA}
\institute{Universidade Federal da Bahia}
\date{Dezembro de 2025}

\begin{document}

%-------------------------------------------------
%  SLIDE DE TÍTULO
%-------------------------------------------------
\begin{frame}[plain,noframenumbering]
    \begin{tikzpicture}[remember picture,overlay]
        % Título principal
        \node[anchor=center] at ([yshift=3.2cm]current page.center) {
            {\fontsize{26}{30}\selectfont\bfseries Sistema de Monitoria-IC}
        };
        % Subtítulo
        \node[anchor=center] at ([yshift=2.2cm]current page.center) {
            {\fontsize{12}{15}\selectfont Plataforma Web para Gestão Completa de Monitorias Acadêmicas da UFBA}
        };
        % Autor
        \node[anchor=center] at ([yshift=-0.8cm]current page.center) {
            {\fontsize{14}{17}\selectfont\bfseries Luis Felipe Cordeiro Sena}
        };
        % Orientador
        \node[anchor=center] at ([yshift=-1.6cm]current page.center) {
            {\fontsize{11}{14}\selectfont Orientador: Prof. Frederico Araújo Durão}
        };
        % Instituição
        \node[anchor=center] at ([yshift=-2.6cm]current page.center) {
            {\fontsize{9}{11}\selectfont Universidade Federal da Bahia}
        };
        % Data
        \node[anchor=center] at ([yshift=-3.2cm]current page.center) {
            {\fontsize{9}{11}\selectfont Dezembro de 2025}
        };
    \end{tikzpicture}
\end{frame}

%-------------------------------------------------
%  SLIDE DE AGENDA
%-------------------------------------------------
\begin{frame}{Agenda}
    \begin{table}
        \begin{tabularx}{0.95\linewidth}{|l|X|}
            \hline
            \textbf{Parte} & \textbf{Conteúdo} \\
            \hline
            1 & \textbf{Introdução} -- Motivação, problema identificado e objetivos \\
            \hline
            2 & \textbf{Fundamentação} -- Monitoria acadêmica e Sistemas de Informação \\
            \hline
            3 & \textbf{Estado da Prática} -- Levantamento em universidades brasileiras \\
            \hline
            4 & \textbf{O Sistema} -- Arquitetura, tecnologias e fluxo de processos \\
            \hline
            5 & \textbf{Avaliação e Conclusão} -- Resultados e trabalhos futuros \\
            \hline
        \end{tabularx}
    \end{table}
\end{frame}

\setlength{\parskip}{0.8em}

%=================================================
\section{Introdução}
%=================================================

\begin{frame}{Motivação}
    \begin{columns}[T]
        \begin{column}{0.52\textwidth}
            \begin{itemize}
                \item \textbf{Monitoria acadêmica:} pilar do ensino superior brasileiro (Lei nº 9.394/96)
                \item \textbf{Benefícios:} desenvolvimento pedagógico, auxílio no ensino-aprendizagem
                \item \textbf{Fluxo complexo:} múltiplos atores e etapas interdependentes
                \item \textbf{Realidade atual:} processos manuais e fragmentados
            \end{itemize}
        \end{column}
        \begin{column}{0.40\textwidth}
            \centering
            \begin{tikzpicture}[scale=0.65, transform shape]
                \node[draw, rounded corners, fill=blue!20, minimum width=2.5cm, minimum height=0.8cm] (mon) at (0,0) {\textbf{Monitoria}};
                \node[draw, rounded corners, fill=green!20, minimum width=2cm, minimum height=0.6cm] (prof) at (-2.5,1.5) {Professor};
                \node[draw, rounded corners, fill=orange!20, minimum width=2cm, minimum height=0.6cm] (aluno) at (2.5,1.5) {Aluno};
                \node[draw, rounded corners, fill=red!20, minimum width=2cm, minimum height=0.6cm] (admin) at (-2.5,-1.5) {Admin};
                \node[draw, rounded corners, fill=purple!20, minimum width=2cm, minimum height=0.6cm] (prograd) at (2.5,-1.5) {PROGRAD};
                \draw[->, thick] (prof) -- (mon);
                \draw[->, thick] (aluno) -- (mon);
                \draw[->, thick] (admin) -- (mon);
                \draw[->, thick] (mon) -- (prograd);
            \end{tikzpicture}
        \end{column}
    \end{columns}
\end{frame}

\begin{frame}{Identificação do Problema}
    \begin{columns}[T]
        \begin{column}{0.46\textwidth}
            \textbf{\textcolor{red!70}{Lado Docente}}
            \begin{itemize}
                \item Retrabalho sistemático (recriação de projetos a cada semestre)
                \item Dispersão de documentos (planilhas, PDFs, e-mails)
                \item Ausência de trilhas de auditoria
            \end{itemize}
            \vspace{0.3em}
            \textbf{\textcolor{orange!70}{Lado Discente}}
            \begin{itemize}
                \item Descoberta de vagas imprevisível
                \item Jornada fragmentada
                \item Opacidade no processo
            \end{itemize}
        \end{column}
        \begin{column}{0.46\textwidth}
            \textbf{\textcolor{blue!70}{Administrativo}}
            \begin{itemize}
                \item Consolidação de dados heterogêneos
                \item Dificuldade de conformidade com prazos
                \item Relatórios custosos
            \end{itemize}
            \vspace{0.5cm}
            \centering
            \fbox{\parbox{0.9\linewidth}{\centering\small\textbf{Lacuna:} Falta de sistema \textit{fim-a-fim} específico para monitoria}}
        \end{column}
    \end{columns}
\end{frame}

\begin{frame}{Objetivos}
    \textbf{Objetivo Central:} Desenvolver plataforma Web completa para gestão de monitorias do IC-UFBA
    \vspace{0.5cm}
    \begin{columns}[T]
        \begin{column}{0.46\textwidth}
            \begin{enumerate}
                \item Digitalizar ciclo completo de projetos
                \item Automatizar processo seletivo
                \item Sistematizar alocação de bolsas
            \end{enumerate}
        \end{column}
        \begin{column}{0.46\textwidth}
            \begin{enumerate}
                \setcounter{enumi}{3}
                \item Eliminar trabalhos manuais repetitivos
                \item Fornecer base analítica para decisões
            \end{enumerate}
        \end{column}
    \end{columns}
    \vspace{0.5cm}
    \centering
    \textit{Substituir fluxos manuais por sistema único com transparência, rastreabilidade e padronização}
\end{frame}

%=================================================
\section{Fundamentação Teórica}
%=================================================

\begin{frame}{Monitoria Acadêmica na UFBA}
    \begin{center}
        \begin{tikzpicture}[scale=0.75, transform shape]
            \draw[thick, ->, >=stealth] (0,0) -- (13,0) node[right] {\small Fluxo};

            \foreach \x/\fase/\col in {
                1/Planejamento/gray!40,
                3.5/Projetos/yellow!40,
                6/Aprovação/orange!40,
                8.5/Seleção/blue!40,
                11/Consolidação/green!40
            } {
                \draw[thick] (\x,0.1) -- (\x,-0.1);
                \node[draw, rounded corners, fill=\col, below, font=\scriptsize, minimum width=1.5cm] at (\x,-0.5) {\fase};
            }
        \end{tikzpicture}
    \end{center}
    \vspace{0.3cm}
    \begin{itemize}
        \item \textbf{Regulamentação:} Lei nº 9.394/96 (LDB) e diretrizes institucionais
        \item \textbf{Critérios:} Nota na disciplina + Coeficiente de Rendimento (CR)
        \item \textbf{Modalidades:} Bolsista e Voluntário
        \item \textbf{Benefícios:} Habilidades didáticas, melhoria de desempenho, apoio docente
    \end{itemize}
\end{frame}

\begin{frame}{Sistemas de Informação: SPT e SIG}
    \begin{columns}[T]
        \begin{column}{0.46\textwidth}
            \textbf{SPT -- Sistema de Processamento de Transações}
            \begin{itemize}
                \item Operações rotineiras e cotidianas
                \item Alta precisão e confiabilidade
                \item Auditoria e rastreabilidade
                \item Alta disponibilidade
            \end{itemize}
            \vspace{0.3cm}
            \small\textit{``Sistemas que realizam e registram transações rotineiras necessárias ao funcionamento organizacional''} (Laudon \& Laudon, 2011)
        \end{column}
        \begin{column}{0.46\textwidth}
            \textbf{SIG -- Sistema de Informações Gerenciais}
            \begin{itemize}
                \item Resumos e relatórios consolidados
                \item Análise de tendências históricas
                \item Apoio à tomada de decisão
                \item Distribuição de recursos
            \end{itemize}
            \vspace{0.3cm}
            \centering
            \fbox{\parbox{0.85\linewidth}{\centering\small O Sistema de Monitoria-IC integra \textbf{SPT + SIG}}}
        \end{column}
    \end{columns}
\end{frame}

%=================================================
\section{Estado da Prática}
%=================================================

\begin{frame}{Levantamento em Universidades Brasileiras}
    \small
    \begin{center}
        \begin{tabularx}{0.92\linewidth}{|l|c|X|}
            \hline
            \textbf{Universidade} & \textbf{Sistema} & \textbf{Observação} \\
            \hline
            USP & Júpiter & Formulários complementares \\
            \hline
            UFRJ & SIGA & Documentos administrativos \\
            \hline
            UnB & SIGAA & Procedimentos departamentais \\
            \hline
            UFSC & CAGR & Editais e orientações externas \\
            \hline
            UNIFESP & SEI & Tramitação de documentos \\
            \hline
        \end{tabularx}
    \end{center}
    \vspace{0.3cm}
    \begin{itemize}
        \item \textbf{Padrão observado:} Sistemas acadêmicos genéricos + processos manuais
        \item \textbf{Fragmentação:} Formulários, e-mails, planilhas dispersas
        \item \textbf{Lacuna:} Nenhum módulo específico e completo para monitoria identificado
    \end{itemize}
\end{frame}

\begin{frame}{Comparação com Sistemas Existentes}
    \scriptsize
    \begin{center}
        \begin{tabularx}{0.95\linewidth}{|l|c|c|c|}
            \hline
            \textbf{Sistema} & \textbf{Workflow fim-a-fim} & \textbf{Específico monitoria} & \textbf{Automação} \\
            \hline
            JúpiterWeb (USP) & Não & Não & Parcial \\
            \hline
            SIGA (UFRJ) & Não & Não & Parcial \\
            \hline
            SIGAA (UnB) & Parcial & Não & Parcial \\
            \hline
            CAGR (UFSC) & Não & Não & Parcial \\
            \hline
            SEI (UNIFESP) & Não & Não & Não \\
            \hline
            \rowcolor{green!20}
            \textbf{Monitoria-IC} & \textbf{Sim (fases 1-5)} & \textbf{Sim} & \textbf{Sim} \\
            \hline
        \end{tabularx}
    \end{center}
    \vspace{0.3cm}
    \begin{itemize}
        \item \textbf{Workflow fim-a-fim:} Cobertura completa das fases 1-5 do ciclo de monitoria
        \item \textbf{Específico monitoria:} Módulo dedicado vs funcionalidade genérica
        \item \textbf{Automação:} PDF, e-mail, assinaturas, equivalências automáticas
    \end{itemize}
\end{frame}

\begin{frame}{Diferenciais do Sistema de Monitoria-IC}
    \begin{center}
        \begin{tikzpicture}[scale=0.75, transform shape,
            box/.style={draw, rounded corners, minimum width=3cm, minimum height=1.2cm, align=center, font=\small, drop shadow}]

            \node[box, fill=blue!25] (esp) at (0,1.5) {\textbf{Específico}\\para monitoria};
            \node[box, fill=green!25] (ciclo) at (4,1.5) {\textbf{Ciclo completo}\\fases 1-5};
            \node[box, fill=orange!25] (auto) at (8,1.5) {\textbf{Automação}\\PDF, e-mail, assinaturas};
            \node[box, fill=purple!25] (stack) at (2,-0.8) {\textbf{Stack moderno}\\Next.js, tRPC, PostgreSQL};
            \node[box, fill=red!25] (audit) at (6,-0.8) {\textbf{Auditabilidade}\\RBAC, histórico};
        \end{tikzpicture}
    \end{center}
\end{frame}

%=================================================
\section{O Sistema}
%=================================================

\begin{frame}{Arquitetura em Camadas}
    \begin{center}
        \begin{tikzpicture}[scale=0.7, transform shape,
            layer/.style={rounded corners, draw=gray!60, fill=yellow!10, thick, inner sep=8pt},
            box/.style={draw=gray!70, rounded corners, fill=yellow!6, text width=3.5cm, align=center, minimum height=0.8cm, font=\scriptsize\bfseries}]

            % Camada de Apresentação
            \node[box, fill=blue!15] (pres) at (-4,2) {Apresentação\\Next.js + React};

            % Camada de Aplicação
            \node[box, fill=green!15] (app) at (0,2) {Aplicação\\tRPC + Lucia Auth};

            % Camada de Negócio
            \node[box, fill=orange!15] (neg) at (4,2) {Negócio\\Services + Repositories};

            % Camada de Dados
            \node[box, fill=purple!15] (data) at (0,0) {Dados\\PostgreSQL + MinIO + Drizzle ORM};

            % Conectores
            \draw[->, thick] (pres) -- (app);
            \draw[->, thick] (app) -- (neg);
            \draw[->, thick] (app) -- (data);
            \draw[->, thick] (neg) -- (data);
        \end{tikzpicture}
    \end{center}
    \vspace{0.3cm}
    \begin{itemize}
        \item \textbf{Separação clara:} Baixo acoplamento, alta coesão
        \item \textbf{Type-safety:} TypeScript em toda a stack
        \item \textbf{Modularidade:} Facilita manutenção e testes
    \end{itemize}
\end{frame}

\begin{frame}{Stack Tecnológico}
    \begin{columns}[T]
        \begin{column}{0.46\textwidth}
            \textbf{Frontend}
            \begin{itemize}
                \item Next.js 15.1.4 (App Router)
                \item TypeScript 5.x
                \item Tailwind CSS + shadcn/ui
                \item React Hook Form + Zod
                \item React Query (TanStack)
            \end{itemize}
        \end{column}
        \begin{column}{0.46\textwidth}
            \textbf{Backend}
            \begin{itemize}
                \item tRPC v11 (API type-safe)
                \item Lucia Auth (sessões)
                \item Drizzle ORM + PostgreSQL
                \item MinIO (armazenamento S3)
                \item Nodemailer (e-mails)
            \end{itemize}
        \end{column}
    \end{columns}
    \vspace{0.5cm}
    \textbf{DevOps:} Docker, GitHub Actions (CI/CD), Vitest, Playwright (E2E)
\end{frame}

\begin{frame}{Fluxo de Processos -- Visão Geral}
    \begin{center}
        \includegraphics[height=0.70\textheight]{process-flow.png}
    \end{center}
    \vspace{0.1cm}
    \scriptsize
    \textbf{Atores:} Admin, Professor, Aluno, Chefe Depto | \textbf{Externos:} Instituto, PROGRAD, NUMOP
\end{frame}

\begin{frame}{Detalhamento das Fases}
    \scriptsize
    \begin{columns}[T]
        \begin{column}{0.32\textwidth}
            \textbf{\textcolor{gray!70}{Fases 1-2: Projetos}}
            \begin{itemize}
                \item Admin importa planilha SIAPE
                \item Professor cria/reutiliza template
                \item Assina digitalmente projeto
                \item Admin aprova/rejeita
                \item Gera planilha para Instituto
            \end{itemize}
        \end{column}
        \begin{column}{0.32\textwidth}
            \textbf{\textcolor{orange!70}{Fase 3: Bolsas \& Edital}}
            \begin{itemize}
                \item PROGRAD informa total bolsas
                \item Admin aloca por projeto
                \item Professor define voluntários
                \item Chefe assina edital
                \item Admin publica e notifica
            \end{itemize}
        \end{column}
        \begin{column}{0.32\textwidth}
            \textbf{\textcolor{blue!70}{Fases 4-5: Seleção}}
            \begin{itemize}
                \item Aluno se inscreve online
                \item Sistema captura CR + notas
                \item Considera equivalências
                \item Professor avalia e seleciona
                \item Aluno aceita + dados bancários
                \item Planilha final p/ PROGRAD
            \end{itemize}
        \end{column}
    \end{columns}
    \vspace{0.4cm}
    \centering
    \fbox{\parbox{0.75\linewidth}{\centering\scriptsize\textbf{Nota:} PROGRAD não acessa o sistema -- comunicação via e-mail/planilhas}}
\end{frame}

\begin{frame}{Funcionalidades por Perfil}
    \begin{columns}[T]
        \begin{column}{0.30\textwidth}
            \textbf{\textcolor{red!70}{Admin}}
            \begin{itemize}
                \item Dashboard métricas
                \item Importar planejamento
                \item Aprovar projetos
                \item Alocar bolsas
                \item Publicar editais
                \item Gerar planilhas
            \end{itemize}
        \end{column}
        \begin{column}{0.30\textwidth}
            \textbf{\textcolor{blue!70}{Professor}}
            \begin{itemize}
                \item Templates reutilizáveis
                \item Assinar projetos (PDF)
                \item Gerenciar candidatos
                \item Avaliar e selecionar
                \item Publicar resultados
            \end{itemize}
        \end{column}
        \begin{column}{0.30\textwidth}
            \textbf{\textcolor{green!70!black}{Estudante}}
            \begin{itemize}
                \item Ver vagas disponíveis
                \item Inscrever-se online
                \item Acompanhar resultados
                \item Aceitar/rejeitar vaga
                \item Preencher dados bancários
            \end{itemize}
        \end{column}
    \end{columns}
\end{frame}

%=================================================
\section{Avaliação e Conclusão}
%=================================================

\begin{frame}{Metodologia de Avaliação com Usuários}
    \textbf{Estudo Qualitativo Exploratório} (N=2)
    \vspace{0.3cm}

    \begin{columns}[T]
        \begin{column}{0.48\textwidth}
            \textbf{Técnicas Utilizadas:}
            \begin{itemize}
                \item Walkthrough guiado
                \item Protocolo Think-Aloud
                \item Entrevista semiestruturada
            \end{itemize}
            \vspace{0.3cm}
            \textbf{Justificativa:}\\
            \small Com N=2, análise quantitativa (Likert) não tem validade estatística. Abordagem qualitativa captura insights ricos.
        \end{column}
        \begin{column}{0.48\textwidth}
            \textbf{Participantes:}
            \begin{itemize}
                \item Prof.\ Rubisley (Admin)
                \item Prof.\ Frederico (Professor)
            \end{itemize}
            \vspace{0.3cm}
            \textbf{Duração:} ~1h por participante\\
            \textbf{Registro:} Gravação tela + áudio
        \end{column}
    \end{columns}
    \vspace{0.4cm}
    \centering
    \small
    \textbf{Validação Técnica:} 55 testes automatizados | 100\% aprovação | 1.07s execução
\end{frame}

\begin{frame}{Roteiro de Avaliação}
    \scriptsize
    \begin{columns}[T]
        \begin{column}{0.32\textwidth}
            \textbf{\textcolor{red!70}{Sessão 1: Admin}}
            \begin{itemize}
                \item Login como admin
                \item Importar planilha DCC
                \item Verificar notificações
                \item Ver projetos criados
            \end{itemize}
        \end{column}
        \begin{column}{0.32\textwidth}
            \textbf{\textcolor{blue!70}{Sessão 2: Professor}}
            \begin{itemize}
                \item Recuperar senha
                \item Completar onboarding
                \item Criar template
                \item Editar projeto
                \item Assinar e submeter
            \end{itemize}
        \end{column}
        \begin{column}{0.32\textwidth}
            \textbf{\textcolor{green!70!black}{Sessão 3: Admin}}
            \begin{itemize}
                \item Ver projeto submetido
                \item Analisar e aprovar
                \item Gerar planilha PROGRAD
            \end{itemize}
        \end{column}
    \end{columns}
    \vspace{0.4cm}
    \textbf{Entrevista Pós-Uso (15 min):}
    \begin{itemize}
        \item Experiência geral, pontos positivos/negativos
        \item Comparação com processo manual atual
        \item Estimativa de economia de tempo
        \item Barreiras para adoção real
        \item Sugestões de melhoria
    \end{itemize}
\end{frame}

\begin{frame}{Resultados da Avaliação Qualitativa}
    \begin{columns}[T]
        \begin{column}{0.48\textwidth}
            \textbf{Perfil Administrador:}
            \begin{itemize}
                \item Pontos positivos: \textit{TODO}
                \item Dificuldades: \textit{TODO}
                \item Sugestões: \textit{TODO}
            \end{itemize}
            \vspace{0.3cm}
            \textbf{Perfil Professor:}
            \begin{itemize}
                \item Pontos positivos: \textit{TODO}
                \item Dificuldades: \textit{TODO}
                \item Sugestões: \textit{TODO}
            \end{itemize}
        \end{column}
        \begin{column}{0.48\textwidth}
            \textbf{Análise Cruzada:}
            \begin{itemize}
                \item Padrões comuns: \textit{TODO}
                \item Divergências: \textit{TODO}
            \end{itemize}
            \vspace{0.3cm}
            \textbf{Economia de Tempo Estimada:}\\
            \textit{TODO} (baseado nas percepções)
            \vspace{0.3cm}
            \textbf{Intenção de Adoção:}\\
            \textit{TODO}
        \end{column}
    \end{columns}
\end{frame}

\begin{frame}{Contribuições do Trabalho}
    \begin{enumerate}
        \item \textbf{Arquitetura especializada:} Três camadas (Router-Service-Repository) para domínio de monitoria com requisitos formais de auditoria

        \item \textbf{Workflow automatizado:} Fases 1-5 do ciclo (templates, assinaturas, equivalências, PDFs) com redução estimada de 80\% em tarefas repetitivas

        \item \textbf{Estudo de caso:} Transformação digital em instituição pública com evidências técnicas (55 testes, 100\% aprovação)

        \item \textbf{Código aberto:} Tecnologias modernas com documentação para replicação em outras universidades
    \end{enumerate}
\end{frame}

\begin{frame}{Limitações e Trabalhos Futuros}
    \begin{columns}[T]
        \begin{column}{0.46\textwidth}
            \textbf{Limitações Atuais}
            \begin{itemize}
                \item Sem integração automática com SIAC (anexo manual de histórico)
                \item Escopo restrito ao IC-UFBA
                \item Validação com usuários finais pendente
                \item Sem aplicativo móvel nativo
            \end{itemize}
        \end{column}
        \begin{column}{0.46\textwidth}
            \textbf{Próximos Passos}
            \begin{itemize}
                \item Piloto institucional (6 meses)
                \item Validação passo a passo c/ usuários
                \item Integração com SIAC/SIGAA
                \item App móvel (React Native)
                \item Expansão para outros departamentos
            \end{itemize}
        \end{column}
    \end{columns}
    \vspace{0.5cm}
    \centering
    \textit{Longo prazo: Generalização para outras universidades públicas brasileiras}
\end{frame}

\begin{frame}{Conclusão}
    \begin{itemize}
        \item O Sistema de Monitoria-IC representa um \textbf{avanço significativo} na modernização da gestão acadêmica

        \item \textbf{Digitalização e automação} de processos tradicionalmente manuais

        \item Base sólida para \textbf{transformação digital contínua} das universidades brasileiras

        \item Demonstração de que é possível criar \textbf{soluções tecnológicas específicas} para problemas acadêmicos complexos

        \item Alto \textbf{potencial de replicação} em outras instituições com desafios similares
    \end{itemize}
    \vspace{0.5cm}
    \centering
    \fbox{\parbox{0.8\linewidth}{\centering\textit{``Substituir fluxos manuais e dispersos por um sistema único que garanta transparência, rastreabilidade e padronização''}}}
\end{frame}

%=================================================
\section{Encerramento}
%=================================================
\begin{frame}{Perguntas?}
    \begin{center}
        \Huge Obrigado!
        \vspace{1cm}

        \normalsize
        \textbf{Luis Felipe Cordeiro Sena}\\
        luis.sena@ufba.br\\
        \vspace{0.5cm}
        \textbf{Orientador:} Prof. Frederico Araújo Durão\\
        fdurao@ufba.br\\
        \vspace{0.5cm}
        \small
        Código disponível em: \url{github.com/luisfelipesena/sistema-de-monitoria-ic}
    \end{center}
\end{frame}

\end{document}
